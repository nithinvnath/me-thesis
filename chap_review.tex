\chapter{Review}
\label{chap:review}

\begin{quote} \small
	We will briefly look at the related works in the Secure Deduplication field
\end{quote}

Douceur et al. \cite{convergentEnc} were the first to propose a novel deterministic cryptosystem
called \textit{convergent encryption} (CE) that enables deduplication to work with
encryption. The idea was to derive the encryption key from the message itself. This
will result in users possessing the same message to end up with the same ciphertext.
\\ \\
Halevi et al. \cite{proofOwner} identified several attacks that exploit client-side
deduplication and introduced the concept of proofs-of-ownership to mitigate these. This
work was extended to get a secure client-side deduplication by Xu et al. in \cite{weakLeakage}.
\\ \\
Bellare et al. in their 2013 paper \cite{mle} formalized the notions of security
in secure deduplication using a new cryptographic primitive called 
\textit{Message-Locked Encryption} (MLE) which subsumes CE. 
The paper was the first to formally argue
the security of secure deduplication and analysed the security of existing schemes as
well as put forward new practical security schemes.
\\ \\
In \cite{imle} Bellare et al. built upon MLE and put forward a new scheme they called
\textit{Interactive Message-Locked Encryption} (iMLE) in which upload and download are
protocols. They modelled privacy and security as games that enabled to argue for stronger notions
of security such as when an adversary controls multiple clients.